\documentclass[12pt]{article}

\usepackage[a4paper, margin=1in]{geometry}

\usepackage[utf8]{inputenc}
\usepackage{float}

\usepackage{amsmath}

\usepackage{circuitikz}


\newcommand{\ov}[1]{\bar{#1}}

\title{Voorbereiding oefeningen les 10}
\author{
        Leen Van Houdt \\
        Sander Mergan \\
        Seppe Duwé \\
        Willem Melis \\
        Wouter Duyols \\
        Xavier Dejager
}

\date{\today}

\begin{document}
\maketitle

\section{Oefening 0.5}

Met een tegenvoorbeeld kunnen we aantonen dat de bewering niet waar is. We stellen dat $x=0$ en $z=0$. Er geldt dat $xy=0$ en $xz=0$. Hieruit volgt dat $xy=xz$. We weten echter dat $y\neq z$. De bewering geldt dus enkel als $x=1$.
\vspace{2mm}
\newline
We kunnen de bewering dus aanvullen met ``gegeven dat $x=1$'', en dit kan bewezen worden door $x$ te vervangen door $1$: $xy = xz \Rightarrow 1.y = 1.z \Rightarrow y = z$ maar dit is uiteraard triviaal.
\vspace{2mm}
\newline
Ook kunnen we de bewering omdraaien: ``in de Booleaanse algebra volgt uit $y=z$ dat $xy=xz$'', en dit kan bewezen worden door $y$ te vervangen door $z$ in de 2\textsuperscript{e} vergelijking maar ook dit bewijs is triviaal.

\clearpage

\section{Oefening 4.30}

\paragraph{Deel 1}
Met de exhaustieve methode (alle gevallen nagaan in de waarheidstabel) vinden we volgende oplossing:
\begin{table}[H]
\centering
\begin{tabular}{|r||c|c|c|c||c|}
	\hline
	\bf{N} & \bf{x} & \bf{y} & \bf{z} & \bf{t} & \bf{v} \\
	\hline
	0    & 0 & 0 & 0 & 0    & 1 \\
	1    & 0 & 0 & 0 & 1    & 0 \\
	2    & 0 & 0 & 1 & 0    & 0 \\
	3    & 0 & 0 & 1 & 1    & 1 \\
	4    & 0 & 1 & 0 & 0    & 0 \\
	5    & 0 & 1 & 0 & 1    & 1 \\
	6    & 0 & 1 & 1 & 0    & 1 \\
	7    & 0 & 1 & 1 & 1    & 0 \\
	8    & 1 & 0 & 0 & 0    & 0 \\
	9    & 1 & 0 & 0 & 1    & 1 \\
	10   & 1 & 0 & 1 & 0    & 1 \\
	11   & 1 & 0 & 1 & 1    & 0 \\
	12   & 1 & 1 & 0 & 0    & 1 \\
	13   & 1 & 1 & 0 & 1    & 0 \\
	14   & 1 & 1 & 1 & 0    & 0 \\
	15   & 1 & 1 & 1 & 1    & 1 \\
	\hline
\end{tabular}
\end{table}

De mintermnormaalvorm wordt gevonden door te kijken naar de kolommen waar $f = 1$ is: 
\vspace{2mm}
\newline
$v = \ov{x}\ov{y}\ov{z}\ov{t} + \ov{x}\ov{y}zt + \ov{x}y\ov{z}t + \ov{x}yz\ov{t} + x\ov{y}\ov{z}t + x\ov{y}z\ov{t} + xy\ov{z}\ov{t} + xyzt$
\vspace{2mm}
De maxtermnormaalvorm wordt gevonden door te kijken naar de kolommen waar $f = 0$ is:
\vspace{2mm}
\newline
$v = (x+y+z+\ov{t}) (x+y+\ov{z}+t) (x+\ov{y}+z+t) (x+\ov{y}+\ov{z}+\ov{t}) (\ov{x}+y+z+t) (\ov{x}+y+\ov{z}+\ov{t}) (\ov{x}+\ov{y}+z+\ov{t}) (\ov{x}+\ov{y}+\ov{z}+t)$
\vspace{2mm}
\newline
De maxtermnormaalvorm bijvoorbeeld kan gerealiseerd worden met volgend poortennetwerk:


\texttt{Insert picture of network here \ldots}
% \begin{figure}[H]
% \begin{center}
% \includegraphics[width=1.0\textwidth]{result_11b.eps}
% \end{center}
% \caption{Convergentie van eigenwaardes op diagonaal}
% \label{result11b}
% \end{figure}

\paragraph{Deel 2}

Gevraagd is de vergelijking $v = 1$ op te lossen met de systematische methode. $v = 1$ als $\ov{v} = 0$.

$\ov{v}$ kan makkelijk bekomen worden door de negatie van de maxtermnormaalvorm te nemen en deze vervolgens te vereenvoudigen met de wet van de Morgan:
\vspace*{2mm}
\newline
$\ov{v} = \neg ((x+y+z+\ov{t}) (x+y+\ov{z}+t) (x+\ov{y}+z+t) (x+\ov{y}+\ov{z}+\ov{t}) (\ov{x}+y+z+t) (\ov{x}+y+\ov{z}+\ov{t}) (\ov{x}+\ov{y}+z+\ov{t}) (\ov{x}+\ov{y}+\ov{z}+t)) = 0$
\vspace*{2mm}
\newline
$\Leftrightarrow \ov{x}\ov{y}\ov{z}t + \ov{x}\ov{y}z\ov{t} + \ov{x}y\ov{z}\ov{t} + \ov{x}yzt + x\ov{y}\ov{z}\ov{t} + x\ov{y}zt + xy\ov{z}t + xyz\ov{t} = 0$
\vspace{2mm}
We beschouwen enkel x als veranderlijke en herschrijven de vergelijking:
\vspace{2mm}
\newline
$x(\ov{y}\ov{z}\ov{t} + \ov{y}zt + y\ov{z}t + yz\ov{t}) + \ov{x}(\ov{y}\ov{z}t + \ov{y}z\ov{t} + y\ov{z}\ov{t} + yzt) = 0$ 
\vspace{2mm}
\newline
Met behulp van de wet van De Morgan kunnen we aantonen dat:
\vspace{2mm}
\newline
$\neg (\ov{y}\ov{z}\ov{t} + \ov{y}zt + y\ov{z}t + yz\ov{t}) = (\ov{y}\ov{z}t + \ov{y}z\ov{t} + y\ov{z}\ov{t} + yzt)$
\vspace{2mm}
\newline
Stel voor de leesbaarheid $q = (\ov{y}\ov{z}\ov{t} + \ov{y}zt + y\ov{z}t + yz\ov{t})$, dan volgt dat 
$\ov{q} = \ov{y}\ov{z}t + \ov{y}z\ov{t} + y\ov{z}\ov{t} + yzt$
\vspace{2mm}
\newline
De vergelijking wordt dan:
\vspace{2mm}
\newline
$xq + \ov{x}\ov{q} = 0$
\vspace{2mm}
\newline
De oplossing voor x is:
\vspace{2mm}
\newline
$x = \ov{q} + \ov{q}\lambda = \ov{q}(1+\lambda) = \ov{q} = \ov{y}\ov{z}t + \ov{y}z\ov{t} + y\ov{z}\ov{t} + yzt = \ov{y}(\ov{z}t + z\ov{t}) + y(\ov{z}\ov{t} + zt)$
\vspace{2mm}
\newline
De voorwaarde voor oplosbaarheid is:
\vspace{2mm}
\newline
$q.\ov{q} = 0$
\vspace{2mm}
\newline
Dit geldt echter altijd dus deze voorwaarde is triviaal. $y$, $z$ en $t$ kunnen dus willekeurig gekozen worden. Stel $y = \lambda_{1}$, $z = \lambda_{2}$ en $t = \lambda_{3}$.
\vspace{2mm}
\newline
We krijgen volgend stelsel:
\vspace{2mm}
\newline
\[  
	\begin{cases}
	    x &= \ov{\lambda_{1}}(\ov{\lambda_{2}}\lambda_{3} + \lambda_{2}\ov{\lambda_{3}})+\lambda_{1}(\ov{\lambda_{2}}\ov{\lambda_{3}}+\lambda_{2}\lambda_{3})\\
	    y &= \lambda_{1}\\
	    z &= \lambda_{2}\\
	    t &= \lambda_{3}\\
  	\end{cases}
\]
\vspace{2mm}
\newline
Als we nu alle waarden doorlopen voor  $\lambda_{1}$, $\lambda_{2}$ en $\lambda_{3}$ krijgen we alle oplossingen voor $\ov{v} = 1$ en dus voor $v = 0$.

\begin{table}[H]
\centering
\begin{tabular}{|c|c|c||c|c|c|c|}
	\hline
	\bf{$\lambda_{1}$} & \bf{$\lambda_{2}$} & \bf{$\lambda_{3}$} & \bf{$x$} & \bf{$y$} & \bf{$z$} & \bf{$t$} \\
	\hline
	0 & 0 & 0   &   0 & 0 & 0 & 0 \\
	0 & 0 & 1   &   1 & 0 & 0 & 1 \\
	0 & 1 & 0   &   1 & 0 & 1 & 0 \\
	0 & 1 & 1   &   0 & 0 & 1 & 1 \\
	1 & 0 & 0   &   1 & 1 & 0 & 0 \\
	1 & 0 & 1   &   0 & 1 & 0 & 1 \\
	1 & 1 & 0   &   0 & 1 & 1 & 0 \\
	1 & 1 & 1   &   1 & 1 & 1 & 1 \\
	\hline
\end{tabular}
\end{table}

Als we deze tabel vergelijken met de tabel van de exhaustieve methode (zie deel 1) dan zien we dat deze waarden voor $x$, $y$, $z$ en $t$ inderdaad overeenkomen met $v=1$.

\section{Oefening 4.41}
\paragraph{1}

\paragraph{2}

\paragraph{3}

\paragraph{4}

\end{document}