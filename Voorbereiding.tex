\documentclass[12pt]{article}

\usepackage[a4paper, margin=1in]{geometry}

\usepackage[utf8]{inputenc}
\usepackage{float}

\usepackage{pifont}

\usepackage[parfill]{parskip}

\usepackage{amsmath}
\usepackage{amsfonts}

\usepackage{circuitikz}


\newcommand{\ov}[1]{\bar{#1}}

\title{Voorbereiding oefeningen les 10}
\author{
        Leen Van Houdt \\
        Sander Mergan \\
        Seppe Duwé \\
        Willem Melis \\
        Wouter Duyols \\
        Xavier Dejager
}

\date{\today}

\begin{document}
\maketitle

\section{Oefening 0.5}

Met een tegenvoorbeeld kunnen we aantonen dat de bewering niet waar is. We stellen dat $x=0$ en $z=0$. Er geldt dat $xy=0$ en $xz=0$. Hieruit volgt dat $xy=xz$. We weten echter dat $y\neq z$. De bewering geldt dus enkel als $x=1$.
\vspace{2mm}
\newline
We kunnen de bewering dus aanvullen met ``gegeven dat $x=1$'', en dit kan bewezen worden door $x$ te vervangen door $1$: $xy = xz \Rightarrow 1.y = 1.z \Rightarrow y = z$ maar dit is uiteraard triviaal.
\vspace{2mm}
\newline
Ook kunnen we de bewering omdraaien: ``in de Booleaanse algebra volgt uit $y=z$ dat $xy=xz$'', en dit kan bewezen worden door $y$ te vervangen door $z$ in de 2\textsuperscript{e} vergelijking maar ook dit bewijs is triviaal.

\clearpage

\section{Oefening 4.30}

\paragraph{Deel 1}
Met de exhaustieve methode (alle gevallen nagaan in de waarheidstabel) vinden we volgende oplossing:
\begin{table}[H]
\centering
\begin{tabular}{|r||c|c|c|c||c|}
	\hline
	\bf{N} & \bf{x} & \bf{y} & \bf{z} & \bf{t} & \bf{v} \\
	\hline
	0    & 0 & 0 & 0 & 0    & 1 \\
	1    & 0 & 0 & 0 & 1    & 0 \\
	2    & 0 & 0 & 1 & 0    & 0 \\
	3    & 0 & 0 & 1 & 1    & 1 \\
	4    & 0 & 1 & 0 & 0    & 0 \\
	5    & 0 & 1 & 0 & 1    & 1 \\
	6    & 0 & 1 & 1 & 0    & 1 \\
	7    & 0 & 1 & 1 & 1    & 0 \\
	8    & 1 & 0 & 0 & 0    & 0 \\
	9    & 1 & 0 & 0 & 1    & 1 \\
	10   & 1 & 0 & 1 & 0    & 1 \\
	11   & 1 & 0 & 1 & 1    & 0 \\
	12   & 1 & 1 & 0 & 0    & 1 \\
	13   & 1 & 1 & 0 & 1    & 0 \\
	14   & 1 & 1 & 1 & 0    & 0 \\
	15   & 1 & 1 & 1 & 1    & 1 \\
	\hline
\end{tabular}
\end{table}

De mintermnormaalvorm wordt gevonden door te kijken naar de kolommen waar $f = 1$ is: 

\begin{equation}
v = \ov{x}\ov{y}\ov{z}\ov{t} + \ov{x}\ov{y}zt + \ov{x}y\ov{z}t + \ov{x}yz\ov{t} + x\ov{y}\ov{z}t + x\ov{y}z\ov{t} + xy\ov{z}\ov{t} + xyzt
\end{equation}

De maxtermnormaalvorm wordt gevonden door te kijken naar de kolommen waar $f = 0$ is:

\begin{align}
v = (x+y+z+\ov{t}) (x+y+\ov{z}+t) (x+\ov{y}+z+t) (x+\ov{y}+\ov{z}+\ov{t}) (\ov{x}+y+z+t) \nonumber \\ (\ov{x}+y+\ov{z}+\ov{t}) (\ov{x}+\ov{y}+z+\ov{t}) (\ov{x}+\ov{y}+\ov{z}+t)
\end{align}

De maxtermnormaalvorm bijvoorbeeld kan gerealiseerd worden met volgend poortennetwerk:


\begin{figure}[H]
\begin{center}
\includegraphics[width=1.0\textwidth]{netwerk}
\end{center}
\caption{Poortnetwerk met maxtermen}
\label{result11b}
\end{figure}

\paragraph{Deel 2}

Gevraagd is de vergelijking $v = 1$ op te lossen met de systematische methode. $v = 1$ als $\ov{v} = 0$.

$\ov{v}$ kan makkelijk bekomen worden door de negatie van de maxtermnormaalvorm te nemen en deze vervolgens te vereenvoudigen met de wet van de Morgan:

\begin{align}
\ov{v} = \neg ((x+y+z+\ov{t}) (x+y+\ov{z}+t) (x+\ov{y}+z+t) (x+\ov{y}+\ov{z}+\ov{t}) (\ov{x}+y+z+t) \nonumber \\ (\ov{x}+y+\ov{z}+\ov{t}) (\ov{x}+\ov{y}+z+\ov{t}) (\ov{x}+\ov{y}+\ov{z}+t)) = 0
\end{align}

\begin{equation}
\Leftrightarrow \ov{x}\ov{y}\ov{z}t + \ov{x}\ov{y}z\ov{t} + \ov{x}y\ov{z}\ov{t} + \ov{x}yzt + x\ov{y}\ov{z}\ov{t} + x\ov{y}zt + xy\ov{z}t + xyz\ov{t} = 0
\end{equation}

We beschouwen enkel x als veranderlijke en herschrijven de vergelijking:

\begin{equation}
x(\ov{y}\ov{z}\ov{t} + \ov{y}zt + y\ov{z}t + yz\ov{t}) + \ov{x}(\ov{y}\ov{z}t + \ov{y}z\ov{t} + y\ov{z}\ov{t} + yzt) = 0
\end{equation}

Met behulp van de wet van De Morgan kunnen we aantonen dat:

\begin{equation}
\neg (\ov{y}\ov{z}\ov{t} + \ov{y}zt + y\ov{z}t + yz\ov{t}) = (\ov{y}\ov{z}t + \ov{y}z\ov{t} + y\ov{z}\ov{t} + yzt)
\end{equation}

Stel voor de leesbaarheid $q = (\ov{y}\ov{z}\ov{t} + \ov{y}zt + y\ov{z}t + yz\ov{t})$, dan volgt dat: 
$\ov{q} = \ov{y}\ov{z}t + \ov{y}z\ov{t} + y\ov{z}\ov{t} + yzt$.
De vergelijking wordt dan:

\begin{equation}
xq + \ov{x}\ov{q} = 0
\end{equation}

De oplossing voor x is:

\begin{equation}
x = \ov{q} + \ov{q}\lambda = \ov{q}(1+\lambda) = \ov{q} = \ov{y}\ov{z}t + \ov{y}z\ov{t} + y\ov{z}\ov{t} + yzt = \ov{y}(\ov{z}t + z\ov{t}) + y(\ov{z}\ov{t} + zt)
\end{equation}

De voorwaarde voor oplosbaarheid is:

\begin{equation}
q.\ov{q} = 0
\end{equation}

Dit geldt echter altijd dus deze voorwaarde is triviaal. $y$, $z$ en $t$ kunnen dus willekeurig gekozen worden. Stel $y = \lambda_{1}$, $z = \lambda_{2}$ en $t = \lambda_{3}$.
We krijgen volgend stelsel:

\begin{equation} 
	\begin{cases}
	    x &= \ov{\lambda_{1}}(\ov{\lambda_{2}}\lambda_{3} + \lambda_{2}\ov{\lambda_{3}})+\lambda_{1}(\ov{\lambda_{2}}\ov{\lambda_{3}}+\lambda_{2}\lambda_{3})\\
	    y &= \lambda_{1}\\
	    z &= \lambda_{2}\\
	    t &= \lambda_{3}\\
  	\end{cases}
\end{equation}

Als we nu alle waarden doorlopen voor  $\lambda_{1}$, $\lambda_{2}$ en $\lambda_{3}$ krijgen we alle oplossingen voor $\ov{v} = 1$ en dus voor $v = 0$.

\begin{table}[H]
\centering
\begin{tabular}{|c|c|c||c|c|c|c|}
	\hline
	\bf{$\lambda_{1}$} & \bf{$\lambda_{2}$} & \bf{$\lambda_{3}$} & \bf{$x$} & \bf{$y$} & \bf{$z$} & \bf{$t$} \\
	\hline
	0 & 0 & 0   &   0 & 0 & 0 & 0 \\
	0 & 0 & 1   &   1 & 0 & 0 & 1 \\
	0 & 1 & 0   &   1 & 0 & 1 & 0 \\
	0 & 1 & 1   &   0 & 0 & 1 & 1 \\
	1 & 0 & 0   &   1 & 1 & 0 & 0 \\
	1 & 0 & 1   &   0 & 1 & 0 & 1 \\
	1 & 1 & 0   &   0 & 1 & 1 & 0 \\
	1 & 1 & 1   &   1 & 1 & 1 & 1 \\
	\hline
\end{tabular}
\end{table}

Als we deze tabel vergelijken met de tabel van de exhaustieve methode (zie deel 1) dan zien we dat deze waarden voor $x$, $y$, $z$ en $t$ inderdaad overeenkomen met $v=1$.

\newpage

\section{Oefening 4.41}
\paragraph{Deeloefening 1}
	
	\par De inverse van $x^{2}+1 \mod x^{3}+x^{2}+1$ zoeken:

	\begin{table}[H]
	\centering
	\begin{tabular}{ccccccc|ccc}
		$x^{3}$&+&$x^{2}$&+&$0x$&+&$1$&$x^{2}$&+&$1$ \\
		\cline{8-10}
		$x^{3}$&+&$0x^{2}$&+&$x$&+&$1$&$x$&+&$1$ \\
		\cline{1-7}
		 & &$x^{2}$&-&$x$&+&$1$& & & \\
		 & &$x^{2}$&+&$0x$&+&$1$& & & \\
		\cline{3-7}
		 & & & &$x$&+&$0$& & & \\ 

	\end{tabular}
	\end{table}

	\begin{table}[H]
	\centering
	\begin{tabular}{ccc|c}
		$x^{2}$&+&$1$&x \\
		\cline{4-4}
		$x^{2}$& & &x \\
		\cline{1-3}
		 & &$1$& \\
	\end{tabular}
	\end{table}

	\begin{align}
		x^{3}+x^{2}+1 &= (x+1)(x^{2}+1)+x \\
		x^{2}+1 &= x\cdot x+1
	\end{align}

	\begin{align}
		1 &= (x^{2}+1)-x\cdot x \\
		&= (x^{2}+1)-x(x^{3}+x^{2}+1-(x+1)(x^{2}+1))\\
		&= (1+x+x^{2})(x^{2}+1)-x(x^{3}+x^{2}+1)\\
	\end{align}

	\begin{equation}
		1+x+x^{2} = \text{gezochte inverse} 
	\end{equation}

	\begin{align}
	(x^{2}+1)y(x) &= x^{2}+x+1 \mod(x^{3}+x^{2}+1) \\
	\Leftrightarrow y(x) &= (x^{2}+1)^{-1} (x^{2}+x+1) \mod(x^{3}+x^{2}+1) \\
	&= (x^{2}+x+1)(x^{2}+x+1) \mod(x^{3}+x^{2}+1) \\
	&= x^{4}+x^{2}+1 \mod(x^{3}+x^{2}+1) \\
	&= x \mod(x^{3}+x^{2}+1)
	\end{align}

	\begin{table}[H]
	\centering
	\begin{tabular}{ccccccccc|ccccc}
		$x^{4}$&+&$0x^{3}$&+&$x^{2}$&+&$0x$&+&1&$x^{3}$&+&$x^{2}$&+&$1$ \\
		\cline{10-14}
		$0x^{4}$&+&$x^{3}$&+&$0x^{2}$&+&$x$& & &$x$&-&$1$& & \\
		\cline{1-9}
		$0$&-&$x^{3}$&+&$x^{2}$&-&$x$&+&$1$   &&&&& \\
		 &-&$x^{3}$&-&$x^{2}$&+&$0x$&-&$1$   &&&&& \\
		\cline{2-9}
		 & &$0$&+&$0$&+&$x$&+&$0$   &&&&& \\
	\end{tabular}
	\end{table}

	\paragraph{Deeloefening 2}

	\begin{equation}
		\begin{cases}
			2z + 4y = 1 \qquad \qquad \text{\ding{172}} \\
			3z + 2y = 2 \qquad \qquad \text{\ding{173}}
		\end{cases}
	\end{equation}

	\fbox{modulo 7:}

	\begin{align}
		\text{\ding{172}}+2 \cdot \text{\ding{173}} \Rightarrow & z = 5 \mod 7 \\
		\Rightarrow & 3+4y = 1 \mod 7 \\
		\Leftrightarrow & 4y = -2 = 5 \mod 7 \\
		\Leftrightarrow & y = 5 \cdot 4^{-1} = 5 \cdot 2 = 3 \mod 7
	\end{align}

	\fbox{modulo 6:}

	\begin{align}
		\text{\ding{172}}+2 \cdot \text{\ding{173}} \Rightarrow & z = 5 \mod 6 \\
		\Leftrightarrow & z = 5 \cdot 2^{-1} \mod 6
	\end{align}

\par De inverse van 2 bestaat echter niet in $\mathbb{Z}_{6}$, dus het stelsel is niet eenduidig oplosbaar in $\mathbb{Z}_{6}$.

	\paragraph{Deeloefening 3}

	\paragraph{Deeloefening 4}

	\begin{equation}
        \begin{cases}
            3V = 2 \mod 5 \\
            2V = 2 \mod 7 \\
            4V = 0 \mod 11
        \end{cases}
    \end{equation}
    
    \begin{equation}
        \begin{cases}
            V = 4 \mod 5 \\
            V = 8 \mod 7 \\
            V = 0 \mod 11
        \end{cases}
    \end{equation}
    
    \begin{equation}
        \begin{cases}
            V = 4 \mod 5 \\
            V = 1 \mod 7 \\
            V = 0 \mod 11
        \end{cases}
    \end{equation}
    \begin{equation}
        \begin{cases}
            V = a_1 \mod m_1 \\
            V = a_2 \mod m_2 \\
            V = a_3 \mod m_3
        \end{cases}
    \end{equation}
    \begin{equation}
        M = 5.7.11
    \end{equation}
    
    \begin{equation}
        M1 = \frac{M}{m_1} = \frac{5.7.11}{5} = 7.11 = 77
    \end{equation}
    \begin{equation}
        M2 = \frac{M}{m_2} = \frac{5.7.11}{7} = 5.11 = 55
    \end{equation}
    \begin{equation}
        M3 = \frac{M}{m_3} = \frac{5.7.11}{11} = 5.7 = 35
    \end{equation}
    
    \begin{equation}
        N_1.M_1 = 1 \mod m_1
    \end{equation}
    \begin{equation}
        N_1.77 = 1 \mod 5
    \end{equation}
    \fbox{$N_1 = 3$} 
    
    \begin{equation}
        N_2.M_2 = 1 \mod m_2
    \end{equation}
    \begin{equation}
        N_2.55 = 1 \mod 7
    \end{equation}
    \fbox{$N_2=6$}
    
    \begin{equation}
        N_3.M_3 = 1 \mod m_3
    \end{equation}
    \begin{equation}
        N_3.35 = 1 \mod 11
    \end{equation}
    \fbox{$N_3=6$}
    
    \begin{equation}
        x = \sum_{i=1}^{r} = a_i.N_i.M_i = 4.3.77 + 1.6.55 + 0.6.11 = 1254
    \end{equation}

    \paragraph{Deeloefening 5}



\end{document}